\documentclass[12pt,t]{beamer}
\usetheme{cute}


% Font
\usefonttheme{professionalfonts}
\usepackage{unicode-math}
\setmathfont{STIX Two Math}
\usepackage{fontspec}
\setmainfont[%
  Scale=MatchLowercase,
  BoldFont=STIX Two Text Bold,
  ItalicFont=STIX Two Text Italic,
  BoldItalicFont=STIX Two Text Bold Italic,
]{STIX Two Text}
\setsansfont[%
  Scale=MatchLowercase,
  Numbers={Lining},
  BoldFont=TeXGyreHeros-Bold,
  ItalicFont=TeXGyreHeros-Italic,
  BoldItalicFont=TeXGyreHeros-BoldItalic,
]{TeXGyreHeros-Regular}
\setmonofont[
  Scale=MatchLowercase,
]{DejaVu Sans Mono}
% For Japanese
\usepackage{luatexja-fontspec}
\setmainjfont[
  YokoFeatures={JFM=prop},
  CharacterWidth=Proportional,
  Kerning=On,
  Scale=0.8,
  BoldFont=Harano Aji Gothic Bold,
]{Harano Aji Gothic Regular}
\setsansjfont[
  YokoFeatures={JFM=prop},
  CharacterWidth=Proportional,
  Kerning=On,
  Scale=0.8,
  BoldFont=Harano Aji Gothic Bold,
]{Harano Aji Gothic Regular}


% Document Properties
\title[Cute Theme]{Cute Theme for Beamer}
\subtitle{A theme that makes you happy!}
\author[Y.~Taniguchi]{Yuta Taniguchi}
\institute{@yuttieyuttie}
\date{2021-07-22}


\begin{document}


{
  \setbeamertemplate{footline}{}
  \frame{\titlepage}
}
\setcounter{framenumber}{0}


\begin{frame}{Table of Contents}
  \toc
\end{frame}


\section{Introduction}
\begin{frame}{Table of Contents}
  \toc[currentsection]
\end{frame}


\begin{frame}{What's this?}
  \alert{A "Cute" Theme for LaTeX Beamer}
  \begin{itemize}
    \item Cute
  \end{itemize}
\end{frame}


\section{Examples}
\begin{frame}{Table of Contents}
  \toc[currentsection]
\end{frame}


\begin{frame}{Bullet List}
  \begin{itemize}
  \item AAA
  \item BBB
    \begin{itemize}
    \item 111
    \item 222
      \begin{itemize}
      \item xxx
      \item yyy
      \item zzz
      \end{itemize}
    \end{itemize}
  \end{itemize}
\end{frame}


\begin{frame}{Enumeration}
  \begin{enumerate}
  \item Alpha
    \begin{enumerate}
    \item I
    \item II
    \item III
    \end{enumerate}
  \item Beta
  \item Gamma
  \end{enumerate}
\end{frame}


\begin{frame}{Descriptions}
  \begin{description}
  \item[Abc] basic
  \item[Xyz] extra
  \end{description}
\end{frame}


\begin{frame}[fragile]{Math}
  \begin{itemize}
    \item $e^{ix} = \cos x + i \sin x$
    \item $\Gamma(z) = \int_0^\infty t^{z-1} e^{-t} dt$
  \end{itemize}
\end{frame}


\begin{frame}[fragile]{Code}
  \begin{columns}
    \begin{column}{0.5\textwidth}
      \begin{itemize}
        \item Syntax highlighting by \alert{minted} package
      \end{itemize}
    \end{column}
    \begin{column}{0.5\textwidth}
      \begin{minted}[linenos, fontsize=\footnotesize]{python}
def main():
    print('Hello, Beamer!')

if __name__ == '__main__':
    main()
      \end{minted}
    \end{column}
  \end{columns}
\end{frame}


\begin{frame}{Rich Text}
  \begin{itemize}
  \item \textit{Italic} style
  \item \textbf{Bold} style
  \item \textit{\textbf{Bold italic}} style
  \item \alert{Alerted} style
  \end{itemize}
\end{frame}


\begin{frame}[fragile]{Blocks}
  \begin{block}{Generic Block}
    Generally, we can use this one.
  \end{block}

  \begin{itemize}
    \item Item
      \begin{alertblock}{Alert Block}
        You are alerted!
      \end{alertblock}
    \item Item
      \begin{itemize}
        \item Item
          \begin{exampleblock}{Example Block}
            Here is an example of the \verb|exampleblock|.
          \end{exampleblock}
      \end{itemize}
  \end{itemize}
\end{frame}


\end{document}
