\documentclass[t]{beamer}  % Specify 'dvipdfmx' class option if you use it

% Additional packages for Japanese typesetting
\usepackage{ifluatex,ifptex,ifxetex}
% (u)platex
\ifptex
\usepackage{bxdpx-beamer}
\usepackage{pxjahyper}
\fi
% xelatex
\ifxetex
\usepackage{zxjatype}
\fi

% Theme
\usetheme{cute}


% Fonts
\usefonttheme{professionalfonts}

\ifluatex

\usepackage{lmodern}
\usepackage{fontspec}
\setmainfont{Latin Modern Roman}[Scale=MatchLowercase]
\setsansfont{TeX Gyre Heros}[Scale=MatchLowercase]
\setmonofont{DejaVu Sans Mono}[Scale=MatchLowercase]
% For Japanese
\usepackage{luatexja-fontspec}
\setmainjfont{Harano Aji Mincho}[YokoFeatures={JFM=prop},CharacterWidth=Proportional,Kerning=On,Scale=MatchLowercase]
\setsansjfont{Harano Aji Gothic}[YokoFeatures={JFM=prop},CharacterWidth=Proportional,Kerning=On,Scale=MatchLowercase]
\renewcommand{\kanjifamilydefault}{\gtdefault}

\else\ifxetex

\usepackage{lmodern}
\usepackage{fontspec}
\setmainfont{Latin Modern Roman}[Scale=MatchLowercase]
\setsansfont{TeX Gyre Heros}[Scale=MatchLowercase]
\setmonofont{DejaVu Sans Mono}[Scale=MatchLowercase]
% For Japanese
\setCJKmainfont[CharacterWidth=Proportional,Kerning=On,Scale=MatchLowercase]{HaranoAjiMincho}
\setCJKsansfont[CharacterWidth=Proportional,Kerning=On,Scale=MatchLowercase]{HaranoAjiGothic}

\else

\usepackage[T1]{fontenc}
\usepackage{lmodern}
\usepackage{tgheros}
\usepackage{DejaVuSansMono}
% For Japanese
\ifptex
\ifuptex
\usepackage[deluxe,expert,uplatex]{otf}
\else
\usepackage[deluxe,expert]{otf}
\fi
\usepackage[haranoaji]{pxchfon}
\renewcommand{\kanjifamilydefault}{\gtdefault}
\fi

\fi\fi


% Syntax highlighting with minted
\usepackage{minted}
\usemintedstyle{solarized-light}


% Insert Table of Contents at the beginning of every section
\AtBeginSection[]{%
  \begin{frame}{Table of Contents}
    \centering
    \toc[currentsection]
  \end{frame}
}


% Document Properties
\title[Cute Theme]{Cute Theme for Beamer}
\subtitle{A theme that makes you happy!}
\author[Y.~Taniguchi]{Yuta Taniguchi}
\institute{@yuttieyuttie}
\date{2021-07-22}


\begin{document}


% Title page
{
  \setbeamertemplate{footline}{}
  \frame{\titlepage}
}
\setcounter{framenumber}{0}


\begin{frame}{Table of Contents}
  \centering
  \toc
\end{frame}


\section{Introduction}
\begin{frame}{What's this?}
  This is a cute theme for LaTeX Beamer.
\end{frame}


\section{Examples}
\begin{frame}{Bullet List}
  \begin{itemize}
    \item AAA
      \begin{itemize}
        \item 111
          \begin{itemize}
            \item XXX
            \item YYY
          \end{itemize}
        \item 222
      \end{itemize}
    \item Lorem ipsum dolor sit amet, consectetur adipiscing elit. Praesent at arcu eget eros sagittis aliquam. Nullam convallis finibus massa eget laoreet.
    \item Interdum et malesuada fames ac ante ipsum primis in faucibus. Aliquam ac lorem sit amet sem viverra vulputate. Donec scelerisque, mi quis egestas ornare, erat ex volutpat urna, a convallis velit tellus id nisi.
    \item Sed gravida tortor eu facilisis cursus. Nullam quam odio, hendrerit ac fringilla id, lobortis eu erat. Donec lobortis pulvinar nibh eu pellentesque.
  \end{itemize}
\end{frame}


\begin{frame}{Enumeration}
  \begin{enumerate}
    \item Alpha
      \begin{enumerate}
        \item I
          \begin{enumerate}
            \item Beta
            \item Gamma
          \end{enumerate}
        \item II
      \end{enumerate}
    \item Lorem ipsum dolor sit amet, consectetur adipiscing elit. Praesent at arcu eget eros sagittis aliquam. Nullam convallis finibus massa eget laoreet.
    \item Interdum et malesuada fames ac ante ipsum primis in faucibus. Aliquam ac lorem sit amet sem viverra vulputate. Donec scelerisque, mi quis egestas ornare, erat ex volutpat urna, a convallis velit tellus id nisi.
    \item Sed gravida tortor eu facilisis cursus. Nullam quam odio, hendrerit ac fringilla id, lobortis eu erat. Donec lobortis pulvinar nibh eu pellentesque.
  \end{enumerate}
\end{frame}


\begin{frame}{Descriptions}
  \begin{description}
    \item[cute] adjective
      \begin{enumerate}
        \item pretty and attractive
      \end{enumerate}
    \item[theme] noun
      \begin{enumerate}
        \item the subject or main idea in a talk, piece of writing or work of art
        \item (music) a short tune that is repeated or developed in a piece of music
      \end{enumerate}
  \end{description}
\end{frame}


\begin{frame}[fragile]{Math}
  \emph{Euler's formula}, named after Leonhard Euler, is a mathematical formula
  in complex analysis that establishes the fundamental relationship between the
  trigonometric functions and the complex exponential function. Euler's formula
  states that for any real number $x$:
  \begin{align}
    e^{ix} = \cos x + i\sin x,
  \end{align}
  where $e$ is the base of the natural logarithm, $i$ is the imaginary unit,
  and $\cos$ and $\sin$ are the trigonometric functions cosine and sine
  respectively.

  Here are some math equations in a list:
  \begin{itemize}
    \item $e^{ix} = \cos x + i \sin x$
    \item $\Gamma(z) = \int_0^\infty t^{z-1} e^{-t} dt$
  \end{itemize}
\end{frame}


\begin{frame}[fragile]{Source Code}
  Although the theme itself provides nothing for the purpose,
  you can use \alert{minted} package for syntax highlighting.

  \begin{minted}[frame=single,fontsize=\footnotesize]{python}
def main():
    print('Hello, Beamer!')

if __name__ == '__main__':
    main()
  \end{minted}
\end{frame}


\begin{frame}{Font Styles}
  \begin{itemize}
  \item \textit{Italic} text
  \item \textbf{Bold} text
  \item \textit{\textbf{Bold italic}} text
  \item \alert{Alerted} text
  \item \emph{Emphasized} text
  \end{itemize}
\end{frame}


\begin{frame}[fragile]{Blocks}
  \begin{block}{Normal Block}
    This is a basic block.
  \end{block}

  \begin{itemize}
    \item Item
      \begin{alertblock}{Alert Block}
        You are alerted!
      \end{alertblock}
    \item Item
      \begin{itemize}
        \item Item
          \begin{exampleblock}{Example Block}
            Here is an example of the \verb|exampleblock|.
          \end{exampleblock}
      \end{itemize}
  \end{itemize}
\end{frame}


\end{document}
