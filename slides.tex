\documentclass[12pt,t,hyperref={pdfencoding=auto}]{beamer}
\usepackage[utf8]{inputenc}
\usepackage[T1]{fontenc}
\usetheme{strong}


% Packages
\usepackage{latexsym}
\usepackage{amssymb,amsmath}
\usepackage{graphicx}
\usepackage{hyperref}
%\usepackage{algorithm}  % must be behind hyperref


% Font
\usefonttheme[onlymath]{serif}  % Use 'serif' font for math
\usepackage{fontspec}
\setsansfont[%
  Scale=MatchLowercase,
  Numbers={Proportional,Lining},
  BoldFont=Lato Regular,
  ItalicFont=Lato Light Italic,
  BoldItalicFont=Lato Italic]
  {Lato Light}
\setmonofont[Scale=MatchLowercase]{DejaVu Sans Mono}
% For Japanese
\usepackage{luatexja-fontspec}
\setmainjfont[Scale=0.8,BoldFont=Harano Aji Gothic Regular]{Harano Aji Gothic Light}
\setsansjfont[Scale=0.8,BoldFont=Harano Aji Gothic Regular]{Harano Aji Gothic Light}


% Path
%\graphicspath{{image/}}


% Mathmatical Functions
%\DeclareMathOperator{\sin}{sin}
%\DeclareMathOperator{\argmax}{argmax}


% Commands
\newcommand{\mathbs}[1]{{\boldsymbol{#1}}}


% Document Properties
\title[Strong Theme]{Beamer Theme\\--- Strong Theme ---}
\subtitle{A theme designed with contrast in mind}
\author[Y.~Taniguchi]{Yuta Taniguchi}
\institute{@yuttieyuttie}
\date{2013-10-29}


\begin{document}


{
  \setbeamertemplate{footline}{}
  \frame{\titlepage}
}
\setcounter{framenumber}{0}


\begin{frame}{Table of Contents}
  \toc
\end{frame}


\section{Introduction}
\begin{frame}{Table of Contents}
  \toc[currentsection]
\end{frame}


\begin{frame}{What's this?}
  \alert{A "Strong" Theme for LaTeX Beamer}
  \begin{itemize}
    \item Contrast in mind
    \item Cute
  \end{itemize}
\end{frame}


\section{Examples}
\begin{frame}{Table of Contents}
  \toc[currentsection]
\end{frame}


\begin{frame}{Basic Structures: Bullet List}
  \begin{itemize}
  \item AAA
    \begin{itemize}
    \item 111
    \item 222
    \end{itemize}
  \item BBB
  \end{itemize}
\end{frame}


\begin{frame}{Basic Structures: Enumeration}
  \begin{enumerate}
  \item Alpha
    \begin{enumerate}
    \item I
    \item II
    \item III
    \end{enumerate}
  \item Beta
  \item Gamma
  \end{enumerate}
\end{frame}


\begin{frame}{Basic Structures: Descriptions}
  \begin{description}
  \item[Abc] basic
  \item[Xyz] extra
  \end{description}
\end{frame}


\begin{frame}[fragile]{Math \& Code}
  \begin{columns}
    \begin{column}{0.5\textwidth}
      \begin{itemize}
      \item $e^{ix} = \cos x + i \sin x$
      \item $\Gamma(z) = \int_0^\infty t^{z-1} e^{-t} dt$
      \end{itemize}
    \end{column}
    \begin{column}{0.5\textwidth}
      \begin{html*}{gobble=8}
        <!DOCTYPE html>
        <html>
          <head>
            <meta charset="UTF-8" />
            <title>Example of Code</title>
          </head>
          <body>
            Syntax highlight by
            <strong class="package">
              minted
            </strong>!
          </body>
        </html>
      \end{html*}
    \end{column}
  \end{columns}
\end{frame}


\begin{frame}{Rich Text}
  \begin{itemize}
  \item We can use \textit{italic} and \textbf{bold} styles
  \item Furthermore, \textit{\textbf{bold italic}} and \textsc{Small Capital} styles
  \item There is also Beamer's \alert{alerted text}.
  \end{itemize}
\end{frame}


\begin{frame}[fragile]{Blocks}
  \begin{block}{Generic Block}
    Generally, we can use this one.
  \end{block}

  \begin{itemize}
    \item Item
      \begin{alertblock}{Alert Block}
        You are alerted!
      \end{alertblock}
    \item Item
      \begin{itemize}
        \item Item
          \begin{exampleblock}{Example Block}
            Here is an example of the \verb|exampleblock|.
          \end{exampleblock}
      \end{itemize}
  \end{itemize}
\end{frame}


\end{document}
